
\section{Introduction}

% There are a few different strategies we can go with here. We could attack the beyond being there problem, and go for the "thinking about face to face" bit. Alternatively, a more general plea for shifting attention back to synchronicity - asynchronous has its issues, and we still (rightfully) privilege real time encounters. The motivating metaphor here being the weirdly object-centric models of asychronous encounters, not people-centric models.

% Some other related themes:
% - stages
% - channels
% - attention
% - text v audio
% - power relationships / confidence?
% - ???

% The core question is what do my systems encompass and lend themselves more to. Clearly they're all synchronous experiences, not async. So that's fine. 

% Okay, here's what we'll do - leave the sync/async pendulum stuff to the background section as a way of framing the history of system design in this space. Focus on the beyond being there story for the introduction.

Throughout the history of mediated communication, there have been fundamental questions about what role mediated communication systems should play. Early in their development, a clear imperative strategy emerged: computer mediated communication systems should focus on recreating the experience of being face-to-face with another person. In a sense, this strategy advocated for building systems that seemed to be invisible to people using them. \cite{weiser} Success was relatively easy to judge, too. Did it feel like you were face to face with someone or not?

There's an undeniable magic to the pursuit of recreating face-to-face communication at a distance. It is most poetically captured in the famous \"Hole in Space\" \cite{hole_in_space} piece, visually and audibly connecting a storefront in LA and New York in a way that seemed to make distance disappear. This same magic continues to motivate modern communication tools from major companies; Cisco and Apple, among others \cite{commercials}, have played on this utopian vision of a future where distance is no barrier to communication in their commercials for so-called \"telepresence\" tools and mobile phones with cameras, respectively. This vision is not simply aspirational, either. Tools to communicate with physically distant people either with audio alone or with an added video connection play a role in the daily lives of millions of people. This desire to experience \"being there\" with someone else is powerful and compelling. 

It is not, however, the only way to approach this problem. In their famous 1992 paper, Hollan and Stornetta \cite{beyond_being_there} suggest an alternative approach which they named \"beyond beyond being there\". They argue that seeking to recreate the experience of \"being there\" was in a way an abdication of our responsibility as designers that left un-explored an important design space. In particular, they urge us to think not about ways to minimize the experience of mediation in communication, but to look for ways that mediation can add value to interactions. To take this perspective seriously, we need to shift away from a view of face-to-face interaction as being always better than interactions mediated by technology and instead think critically about potential limitations and challenges with face-to-face interaction and potential benefits that mediation can offer. 

% think about listifying this section

Many of these challenges are common sense, even if they are frequently forgotten when people argue for recreating face-to-face experiences. Face-to-face communication requires relatively explicit turn-taking; multiple speakers in a group make them all largely unintelligible. In mediated environments like chat, simultaneous conversation threads can easily co-exist for long periods of time. There are major identity implications to face-to-face communication, namely that it's difficult to conduct any face-to-face communication without revealing significant information about your identity. In mediated contexts, there are techniques ranging from anonymity to pseudonymity to limit identity disclosing information. Participation in non-mediated interactions is ephemeral, while mediated interaction can easily be archived and represented either in context or after the fact. Participation in face-to-face situations can be limited by confidence, but mediated participation is usually substantially disinhibited. \cite{???}

% is this a second order effect?
For a variety of reasons, the power dynamics in social situations are more easily subverted in mediated environments. 

All of these challenges and differences between face-to-face interaction and mediated communication (broadly construed) represent a major opportunity. By designing systems that seek to mitigate the challenges of being face-to-face and maximize the potential benefits of mediation, we can create systems that reach the goal of going \"beyond being there\". 

Part of what's attractive about mediated communication systems is that there is a tremendous variety of ways to design and use them once we set aside a desire to recreate face-to-face interaction. Although in this section I've contrasted mediated communication with face-to-face communication in a way that might imply that mediated communication systems are somehow monolithic and self-similar, the survey of related systems in the section to follow will illustrate the tremendous range of potential systems in this space and demonstrate how thoughtful designs can have widely varying impacts on the experience of communication or collaboration. 

In my work I seek to explore part of the design space identified by Hollan and Stornetta by designing, building, and studying communication systems that go \"beyond being there\". The full space of systems for going \"beyond being there\" is vast and I have focused my work in a number of ways. Although many of the interfaces proposed as being \"beyond being there\" in their original paper are asynchronous systems, I focus exclusively on synchronous experiences -- co-temporal experiences where participants are communicating and using a system at the same time. I focus primarily on medium sized groups and crowds. Although the groups are always interacting co-temporally, they are sometimes all in one physical place, and sometimes geographically distributed. There are two major design strategies that my work focuses on: adding non-verbal communication channels to an experience and creating richer, behavior representations of conversation partners. 

It can be difficult to describe exactly what the contributions of design focused research can be. Is making something that people describe as useful a sufficient outcome, or should we expect more? I conceive of my research as being a kind of design cartographer. Given the design space I outlined earlier, I try to identify and describe compelling and effective design strategies. Beyond the specific contributions of design elements that are valuable, I step back from the specifics of each system and theorize about general themes that are relevant to any sort of design work in the space. In this case, these are themes like stages, performative attention and contribution, norms, power dynamics, and grounding. By contributing in both of these ways, I seek to support designers and researchers working both with modern constraints as well as contribute theoretically to ways of thinking about design of co-temporal mediated experiences that will give us the tools to think about as-yet-unimagined design spaces.

% At that scale, I find that the benefits of being face-to-face are strongest and successful interventions are quite difficult to produce. (Leave the research strategic points for later?)

% talk about general themes + backchannels

In this dissertation proposal, I will start with a survey of relevant related systems, as well as briefly touching on experimental, theoretical, and methodological work that plays a significant role in my work. Then I will describe the arc of my past work with a focus on my final project: Tin Can. Finally, I will describe in general terms the contributions and potential impact of this work.

% need to do a bunch of reference-finding for this section. find people talking about the history of telecommuting, early telephone motivations, telegraph, video conferencing, etc to back this all up. 
% try to dig up that reference where people talk about early claims about telecommuting and teleconferencing...

% TODO the core problem with this introduction is that it never really talks about backchannels or side channels or anything. I specify the domain, but not my primary design strategy. It really should be more about that. Another way to put that is that this is not saying "problem X -> solution Y -> justification Z". I could certainly write something like that, and it might be productive. We'll let this percolate (and maybe get someone in the group to read it and see) and come back to that question. 

\section{Background}

% We'll need to do a little organization here. Obviously, farm the references from the Tin Can Edu CSCW paper + backchan.nl paper. We'll need more, ofc, but it's a start. I suspect there will be some ways to separate out systems that allow communication versus those that simply reflect on a main channel. Maybe it's about whether the system itself is a single channel, multi channel, main channel or side channel? 

% write a methodological section here about why it's useful to study this with design work


\section{Systems in Practice}

\subsection{Meeting Spaces}
\subsection{backchan.nl}
\subsection{Tin Can Classroom}
\subsection{Tin Can Meetings}

\subsection{Methodology}

\section{Contributions}

\section{Timeline}

