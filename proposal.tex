\documentclass{tufte-handout}

%\geometry{showframe}% for debugging purposes -- displays the margins

\usepackage{amsmath}

% Set up the images/graphics package
\usepackage{graphicx}
\setkeys{Gin}{width=\linewidth,totalheight=\textheight,keepaspectratio}
\graphicspath{{graphics/}}

\title{Drew Harry Dissertation Proposal (real title to follow)}
\author[Drew Harry]{Drew Harry}
%\date{24 January 2009}  % if the \date{} command is left out, the current date will be used

% The following package makes prettier tables.  We're all about the bling!
\usepackage{booktabs}

% The units package provides nice, non-stacked fractions and better spacing
% for units.
\usepackage{units}

% The fancyvrb package lets us customize the formatting of verbatim
% environments.  We use a slightly smaller font.
\usepackage{fancyvrb}
\fvset{fontsize=\normalsize}

% Small sections of multiple columns
\usepackage{multicol}

% Provides paragraphs of dummy text
\usepackage{lipsum}

% These commands are used to pretty-print LaTeX commands
\newcommand{\doccmd}[1]{\texttt{\textbackslash#1}}% command name -- adds backslash automatically
\newcommand{\docopt}[1]{\ensuremath{\langle}\textrm{\textit{#1}}\ensuremath{\rangle}}% optional command argument
\newcommand{\docarg}[1]{\textrm{\textit{#1}}}% (required) command argument
\newenvironment{docspec}{\begin{quote}\noindent}{\end{quote}}% command specification environment
\newcommand{\docenv}[1]{\textsf{#1}}% environment name
\newcommand{\docpkg}[1]{\texttt{#1}}% package name
\newcommand{\doccls}[1]{\texttt{#1}}% document class name
\newcommand{\docclsopt}[1]{\texttt{#1}}% document class option name

\begin{document}

\maketitle% this prints the handout title, author, and date

\begin{abstract}
\noindent My awesome abstract.
\end{abstract}

%\printclassoptions

% looks like I'm allowed to have text here, but not going to...

\section{Introduction}\label{sec:introduction}

% There are a few different strategies we can go with here. We could attack the beyond being there problem, and go for the "thinking about face to face" bit. Alternatively, a more general plea for shifting attention back to synchronicity - asynchronous has its issues, and we still (rightfully) privilege real time encounters. The motivating metaphor here being the weirdly object-centric models of asychronous encounters, not people-centric models.

% Some other related themes:
% - stages
% - channels
% - attention
% - text v audio
% - power relationships / confidence?
% - ???

% The core question is what do my systems encompass and lend themselves more to. Clearly they're all synchronous experiences, not async. So that's fine. 

% Okay, here's what we'll do - leave the sync/async pendulum stuff to the background section as a way of framing the history of system design in this space. Focus on the beyond being there story for the introduction.

As researchers first started to build technology to help us communicate with other people at a distance, a clear imperative strategy emerged: computer mediated communication systems should focus on recreating the experience of being face-to-face with another person. The best system, in this model, is one that seems to disappear, in same way the best window makes us feel like there's nothing between us and the other side of the glass.

There's an undeniable magic to the pursuit of recreating face-to-face communication at a distance. This magic is most poetically captured in the famous ``Hole in Space'' \citet{hole_in_space} piece, visually and audibly connecting a storefront in LA and New York in a way that seemed to make distance disappear. This same magic continues to motivate modern communication tools from major companies; Cisco and Apple, among others \citet{commercials}, have played on this utopian vision of a future where distance is no barrier to communication in their commercials for so-called ``telepresence'' tools and mobile phones with cameras, respectively. This vision is not simply aspirational, either. Tools to communicate with physically distant people either with audio alone or with an added video connection play a role in the daily lives of millions of people. This desire to experience ``being there'' with someone else is powerful and compelling. 

It is not, however, the only way to approach this problem. In their famous 1992 paper, Hollan and Stornetta \citet{beyond_being_there} suggest an alternative approach which they named ``beyond beyond being there''. They argue that seeking to recreate the experience of ``being there'' was in a way an abdication of our responsibility as designers that left an important design space un-explored. In particular, they urge us to think not about ways to minimize the experience of mediation in communication, but to look for ways that mediation can add value to interactions. To take this perspective seriously, we need to shift away from a view of face-to-face interaction as being always better than interactions mediated by technology and instead think critically about potential limitations and challenges with face-to-face interaction and potential benefits that mediation can offer. 

% think about listifying this section

Many of these challenges are common sense, even if they are frequently forgotten when people argue for recreating face-to-face experiences. Face-to-face communication requires relatively explicit turn-taking; multiple speakers in a group make them all largely unintelligible. In mediated environments like chat, simultaneous conversation threads can easily co-exist for long periods of time. There are major identity implications to face-to-face communication, namely that it's difficult to conduct any face-to-face communication without revealing significant information about your identity. In mediated contexts, there are techniques ranging from anonymity to pseudonymity to limit identity disclosing information. Participation in non-mediated interactions is ephemeral, while mediated interaction can easily be archived and represented either in context or after the fact. Participation in face-to-face situations can be limited by confidence, but mediated participation tends to be more disinhibited. \citet{???}

% is this a second order effect?
% For a variety of reasons, the power dynamics in social situations are more easily subverted in mediated environments. 

All of these challenges and differences between face-to-face interaction and  mediated communication (broadly construed) represent a major opportunity. In my work, I focus on building systems that add new communication channels to create environments where people have ways to express themselves non-verbally in addition to whatever existing communication channels exist. In some of my work this means building systems to augment face to face situations that are not traditionally mediated at all; in other situations, I look at already-mediated experiences and add new kinds of non-verbal communication options. By adding mediated communication channels to other existing channels, we can focus on the affordances of each channel to let it do what it does best.

% Part of what's attractive about mediated communication systems is that there is a tremendous variety of ways to design and use them once we set aside a desire to recreate face-to-face interaction. Although in this section I've contrasted mediated communication with face-to-face communication in a way that might imply that mediated communication systems are somehow monolithic and self-similar, the survey of related systems in the section to follow will illustrate the tremendous range of potential systems in this space and demonstrate how thoughtful designs can have widely varying impacts on the experience of communication or collaboration. 

I have focused my work in a number of ways. Although many of the interfaces proposed as being ``beyond being there'' in their original paper are asynchronous systems, I focus exclusively on synchronous experiences -- co-temporal experiences where participants are communicating and using a system at the same time. I focus primarily on medium sized groups and crowds. Although the groups are always interacting co-temporally, they are sometimes all in one physical place, and sometimes geographically distributed. 

It can be difficult to describe exactly what the contributions of design focused research can be. In my work, I see two major categories of contributions. First, by building systems and testing them \emph{in situ} I can provide concrete guidance about particular specific design strategies and interfaces were used by users in practice. This is valuable for designers and researchers thinking about how they design non-verbal communication systems in their own work. I also engage with broader theoretical questions about attention, signaling, grounding. By contributing to these larger discourses with specific findings in studies of my own systems, I hope to contribute to broader discussions in computer mediated communication. 

% gotta really settle on what my themes are going to be here. current list is:
%  - attention (solid)
%  - signaling (erm - what is this exactly?)
%  - grounding (I like)
%  - 


% It can be difficult to describe exactly what the contributions of design focused research can be. Is making something that people describe as useful a sufficient outcome, or should we expect more? I conceive of my research as being a kind of design cartographer. Given the design space I outlined earlier, I try to identify and describe compelling and effective design strategies. Beyond the specific contributions of design elements that are valuable, I step back from the specifics of each system and theorize about general themes that are relevant to any sort of design work in the space. In this case, these are themes like stages, performative attention and contribution, norms, power dynamics, and grounding. By contributing in both of these ways, I seek to support designers and researchers working both with modern constraints as well as contribute theoretically to ways of thinking about design of co-temporal mediated experiences that will give us the tools to think about as-yet-unimagined design spaces.

% At that scale, I find that the benefits of being face-to-face are strongest and successful interventions are quite difficult to produce. (Leave the research strategic points for later?)

% talk about general themes + backchannels

In this dissertation proposal, I will start with a survey of relevant related systems, as well as briefly touching on experimental, theoretical, and methodological work that plays a significant role in my work. Then I will describe the arc of my past work with a focus on my final project: Tin Can. Finally, I will describe in general terms the contributions and potential impact of this work.

% need to do a bunch of reference-finding for this section. find people talking about the history of telecommuting, early telephone motivations, telegraph, video conferencing, etc to back this all up. 
% try to dig up that reference where people talk about early claims about telecommuting and teleconferencing...

% TODO the core problem with this introduction is that it never really talks about backchannels or side channels or anything. I specify the domain, but not my primary design strategy. It really should be more about that. Another way to put that is that this is not saying "problem X -> solution Y -> justification Z". I could certainly write something like that, and it might be productive. We'll let this percolate (and maybe get someone in the group to read it and see) and come back to that question. 

\section{Background}

Designing new systems for collaboration and communication, as opposed to studying existing systems, has long been a major stream of HCI and CSCW research. This section will summarize the most salient past work in this area, although little of this work is recent and responsive to the significant shifts in the way people use technology to communicate, collaborate, and play. 

For a variety of reasons, there has been somewhat of a shift in interest away from the kind of co-temporal interfaces and experiences I create towards building and studying systems for asynchronous experiences between much larger numbers of people. The advent of research on mass collaboration systems like Wikipedia \citet{kittur} and ``crowd sourcing'' \citet{bernstein} is part of a larger shift away from what was once the center of gravity of systems research. This shift is a natural response to changes in both technology and the common experience of modern collaborative technology users in a web-oriented world where asynchronous interaction became the norm.

In this work, I argue that we should not neglect the design of co-temporal interfaces and experiences for small groups. We should not think just about how to marshal large numbers of people, but think too about how small groups of people who know each other work, recognizing that much of that work happens face-to-face or co-temporally while geographically distant. It is not effective to treat these interactions as a simple increase in tempo on asynchronous interactions. In co-temporal systems, experiences of presence and understanding how we are perceived (and can control those perceptions) by others are quite important. In asynchronous systems, these issues are minimized; we experience others through their actions on shared objects like documents. In my work, I seek to create richer representations of people and better-support person-person interaction instead of person-document-person interaction. 

To further motivate my work and situate it within the larger space of systems design research, I survey related work in this section. The work is organized into three major themes: adding new communication channels, ways to help people reflect on the participation of themselves and others, and theoretical perspectives.


\subsection{Channels}

The primary focus of my work is on designing systems that add new communication channels and understanding how those channels operate in contrast to existing channels. In this section, I will present related work that addresses some of these questions.

The work most directly related to these questions comes from research into so-called ``backchannels'' in presentation and classroom settings. Yardi describes how a chat-based backchannel operates over a semester in a classroom \citet{Yardi:2006uk}, McCarthy et al describe a similar approach at a conference \citet{mccarthy_digital_2004}. Backchannels can also be considered a potential part of non-event-oriented contexts too, like long-term co-working among small groups. \citet{Huang:2003ef} All of these projects share Tin Can's interest in creating new spaces for group communication, and some of these kinds of systems are integrated into the physical spaces the group inhabits. Backchannels are not just focused on co-located groups, however, and Kellogg (among others) \citet{kellogg_leveraging_2006} \citet{Yankelovich:2005bx} has addressed how text and audio backchannels can operate in distributed contexts. This is the main literature that I hope to contribute to. Although past work has addressed in general terms the different ways people use backchannels, it has not sufficiently explained the complicated issues around channel selection, attention, distraction, and identity. Furthermore, in my work I try to move beyond just adding new text or audio channels by adding other kinds of non-verbal signals.

Much of the work on creating shared media spaces, driven by experiments at PARC in the late 1980s and early 1990s is salient to my work. Although in some cases this work focused on creating new primary channels, researchers became quickly attuned to problems of privacy and attention because such systems always co-exist with face-to-face communication, in much the same way that the systems that I design do. The earliest work at PARC \citet{olsen_bly_portland} focused on creating flexible video connections between offices and conference rooms. Subsequent work focused less on a phone-call-like model where connections are created and ended and shifted towards creating spaces with different affordances. Sometimes these involved connecting multiple individuals together, as in CAVECAT \citet{cavecat}, other times researchers focused on creating a long term persistent video connections in, for instance, common areas of distributed research groups in the VideoWindow project\citet{videowindow}. 

Over time, attention shifted more towards a taking advantage of the possibilities to do more than just create ``being there'' experiences. Some researchers experimented with audio-only spaces \citet{thunderwire}, finding that video was not required to create a sense of connection and space for users, but that the properties of audio did require audio-specific etiquette and coping strategies for the system to be useful. iCom represented a particularly rich design perspective on connecting spaces  \citet{stefan_icom}, recognizing that awareness need not be limited to visual awareness, but can extend to information awareness which can be productively embedded in a media space. This embodies the ``beyond being there'' model best of all the work in this research stream: not just trying to create a transparent window between remote spaces, but making something better than a window could be.


% also cite karrie's 
% (For subsequent, more artistically inclined approaches to this design space, see Karrie's blah blah, get a nice figure in here for that.)


Serendipity also evolved as both an important part of sharing an office environment that was not present with most media space systems. Portholes \citet{portholes} addressed this explicitly by giving people a broader view of remote spaces instead of focusing just on main channel interactions. While my work is not concerned with serendipity, this kind of visual side channel carries important awareness information in much the same way that the side channels in my systems add important contextual information to an interaction. CRUISER \citet{fish_kraut_cruiser} offered non-verbal ways to signal a desire to emulate some of the office hallway etiquette for signaling a desire to drop in and chat informally, without the explicitness of placing a call. The addition of moves like ``cruise'', ``glance'', and ``visit'' are similar in approach to actions in my work like voting in backchan.nl, promoting ideas in Tin Can, or move around the field in Information Spaces. 

Early media space researchers proposed a distinction between ``formal'' systems from ``informal'' systems. \citet{olsen_bly_portland} While most of the work discussed here (and much of my own work) tends towards the informal side of that continuum, there are some formal elements in my work and it's productive to consider related work on the more formal end of that spectrum, too. This formality manifests most strongly in Group Decision Support Systems research. These systems (exemplified by the work of Nunamaker \citet{nunamaker_gdss}) provide prescriptive systems to support particular brainstorming, decision making, outlining, and voting. In the typical GDSS configuration, each participant has their own computer and interact with shared structural data in some way, like submitting a new idea or voting on a proposal. The lack of consistent results in comparative work in this area \citet{dennis_results_summary} illustrates the importance of focused design analysis to contextualize findings; it is not useful to view all brainstorming systems as equivalent and comparable in analysis, and I hope that my work will illustrate how the subtleties in interface and approach can have big impacts on outcomes that help explain some of the contradictory results in past GDSS work.

% go hunting for a desanctis and/or poole piece that's not focusing on AST specifically? also can hit berg if we want, but it feels like a bit of a distraction at the moment.

%, they also produced a nice taxonomy of the kinds of tools that would be useful for distributed collaborative groups: synchronous versus asynchronous communication and open processes versus focused processes, a distinction 


% there's a funny note in the portland paper about how they want to shift away from meeting augmentation to async and task coordination. Funny how times change.

% now summarize. 


% going to want to bring up media equation or whatever that book is called. Cliff Nass. 




% thundewire is just audio, basically a single-channel mumble. not so much about results as describing practices that evolved. 
% portholes is ambient awareness about remote places, not live interaction. cut it?
% videowindow is just like hole in space - audio/video fixed in space

% also mention virtual world stuff? MASSIVE might be worth a quick ref

% "shared media systems"

% - video projects like thunderwire + portholes
% - videowindow


% organize the work in this space 
% projects to talk about:
% - voiceloops
% - backchannel literature
% - social proxies
% - mention conferencing apps
% - Nunamaker
% - zephyr?


% \subsection{Theoretical Perspectives}
% thinking about leaving this out

% stuff from cscw paper:
% systems for reflection
% - second messenger
% - "social mirror" (Karahalios) also bergstrom
% - Meeting Mediator
% 
% systems adding new channels
% - (all the backchan literature: yardi, mccarthy, huang, kellog, yankelovich)
% - do a section on nunameker's work and why it's weird
% - 

% - we'll want to at the very least nod to thinks like voiceloops, and all the audio/video stuff like portholes and thunderwire and that kind of thing. farm my generals reading for that part.
%


% theoretical perspectives
% - practice lens?
% - ethnomethodology? 
% - re-farm wanda's reading list to see what else I can pull from there.

%
% We'll need to do a little organization here. Obviously, farm the references from the Tin Can Edu CSCW paper + backchan.nl paper. We'll need more, ofc, but it's a start. I suspect there will be some ways to separate out systems that allow communication versus those that simply reflect on a main channel. Maybe it's about whether the system itself is a single channel, multi channel, main channel or side channel? 

% write a methodological section here about why it's useful to study this with design work

\subsection{Reflection}

Understanding how we present ourselves to others has been a topic of sociological inquiry for quite some time. Although many of the insights of scholars like Goffman \citet{goffman_presentation_of_self} \citet{someone_else} about how we communicate and interpret information about who we are and how we want to be treated are still relevant, what information is available about people has changed substantially. In some of the examples in this section, designers have added some new bit of information about people to a face to face discussion; in others, we don't have any of the traditional information we would get from being face to face with someone and rely on new types of signals to create a sense of people around us. Part of what sets mediated communication apart is the ability to accumulate behavioral histories and represent and reflect those histories to ourselves and others.

My work is substantially inspired by the work of Joan DiMicco on the Second Messenger project. \citet{second_messenger} In this project, participants in a group discussion were presented with a constantly-updating bar-chart visualization representing the relative amount of time they had talked during the discussion. They found that while people who participated a lot without a visualization tended to moderate their participation when the visualization was present, people with low participation did not participate more just because others were participating less. Meeting Mediator \citet{meeting_mediator} took a similar approach, but focused on situations where groups of two people could see each other and had to interact with another group of two people who they could only hear. Using a different visualization, Kim found that groups were more interactive with the system than without, although there was not a correlation with group performance. 

Bergstrom has done a series of projects that adopt a similar design strategy. Conversation Clusters pulls topics from an audio conversation and presents them in clusters on a table-top display. \citet{conv_clusters} Conversation Clock, like Second Messenger and Meeting Mediator, visualizes conversation participation, but uses a timeline metaphor instead of a aggregative metaphor. \citet{conv_clock} Conversation Votes lets uses discreetly vote about the progress of a discussion, and displays anonymous votes on a table-based display.\citet{conv_votes} Karahalios describes this design space as ``social mirrors''. \citet{social_mirrors}

% sneak in a \citet{visiphone} here?

These are all examples of the accumulate and reflect design strategy, where the system tracks some aspect of behavior: verbal participation in the case of Second Messenger and Meeting Mediator, discussion topics and group attitudes in the case of Bergstrom's work. The systems then present that information back to the individual or group and hope is they will use it to reflect on and potentially adjust their behavior. Much of my work uses this strategy in different ways as a way of making non-verbal actions persistent and visual in a way that makes them more visible and long-lasting than momentary events.  

% cite things like last.fm, goodreads, etc in this space?


% \subsection{Identity}

% maybe this whole section is dumb.
% how often do I really deal with this? It's clearly part of the second life work, but not at all part of backchannl, and only a little a part of the Tin Can series. Erm. Move on for now and double back. 

% Many other projects have focused on how peoples' identities are presented. In most of these pieces, there is no face-to-face element, which drives the researcher's interest in developing compelling alternative options that can richly communicate who someone is in a mediated space. Many informal and practical options are in common use; displaying an icon or image chosen by someone and a pseudonym to represent themselves is a widely used design strategy in social applications throughout the internet. These spaces for self expression are frequently augmented by systems that aggregate someone's behavior in that space. This is much like the reflection techniques, but are usually summarizing events beyond any single person's experience. The community site StackOverflow \citet{stack_overflow} provides a particularly rich example of this strategy.

% screenshot a wow forum, stackoverflow, 

% These techniques feel thin compared to the richness of face-to-face interaction, and researchers have developed a number of alternative approaches. Donath's work on data portraiture in a variety of 



% link to things like Ros' work in terms of augmenting face to face interaction with extra information? Or go back to steve mann's cyborg stuff?

% need a paragraph here that's more about the identity side of things. not sure what that will be. chat circles, perhaps? talking in circles? (whichever one had that intersting voting shapes thing)

% how to fit in social proxies and social translucense

% I could totally write this, but I think I'll just defer it until the information spaces section since it's not generally relevant.
% \subsection{Virtual Spaces}

% For a period, virtual environments seemed to present a credible 


\section{Channels in Practice: Early Work}

I have explored my primary theme of adding communication channels to synchronous communication through a series of design research projects. In each piece, I focus on different contexts, group sizes, and needs. In this section, I will describe two major past projects: \emph{Information Spaces} and \emph{backchan.nl}. This shows the development of my work in this space, and points towards my pair of major final projects.

% talk about approach here: why design things? what do we get out of designing things? what questions do we hope to answer? how does this compare to other strategies?

\subsection{Information Spaces}

% TODO Add figures for all of this. Will want lotsssss.
% think about where to add a ref to our chairs piece in that book.
% Also add a bit that calls back to our major themes. History is about grounding, movement and visualization is about signaling. 

% Consider dropping this whole section? Not sure if it's important/relevant/if I have space for it.
For a period, free form virtual worlds \sidenote{As opposed to game oriented worlds like \emph{World of Warcraft}, which are still the only major long-term commercially successful worlds.} like \emph{Second Life} seemed like they might present a credible alternative to traditional video and audio conferencing. Researchers found that people replicated certain face-to-face social conventions in virtual worlds, for instance maintaining similar interpersonal distances as a function of gender as one might in the physical world \citet{yee_interpersonal_distance} or reacting to avatar height in the same ways people do when face-to-face \citet{balison_height_or_whatever}. Furthermore, the rich signals from body language, fashion, and gaze might more easily translate to environments where we are embodied by three dimensional avatars than other alternative systems. In response, residents of \emph{Second Life} and large corporate entities like IBM \citet{something_about_ibm_in_sl} created elaborate virtual office spaces.

Ultimately, this approach was simply recreating the ``being there'' approach by creating virtual spaces that looked like their physical world alternatives and assuming that avatars would be effective replacements for real bodies and failed to recognize the concrete interface challenges with interacting in a virtual world, as well as the potential benefits to working in a virtual world. In \emph{Information Spaces}, I reconsidered what a virtual world meeting space might look like and how it might give meeting participants ways to communicate with each other that they wouldn't have in face-to-face meetings. \citet{CHI_infospace_paper}

The \emph{Information Spaces} meeting space evokes a very abstract sports stadium. The floor of the field is a colored continuum between ``Agree'' and ``Disagree''. An alternative version of the system used ``Keep Talking'' and ``Move On'' to make the tool useful in discussions that weren't composed of decision-making tasks. Avatars can use their position in the space to indicate how they feel about the decision currently being discussed. Surrounding the colored field is a sloped space, like the stands in a stadium, where avatars can stand and be seen as participants without taking a stand on the issue at hand.

This simple mechanic -- where you stand in the space non-verbally communicates your feelings about the meeting -- forms a nice foundation for what a virtual meeting room could look like. It uses the physical form of the space to communicate how it's supposed to be used, without recreating the architecture of a traditional meeting room. The problem is that movements in this space are ephemeral. If we want someone's position to be meaningful, we must highlight salient features of avatar movement like movement without the space, who's speaking, how long someone has stood somewhere, and so on. \emph{Information Spaces} provides these features through in-context visualizations.

These different social utilities are summarized in figure \ref{fig:info_spaces_utilities}. When avatars stand still, a dwell indicator appears underneath their avatar which grows slowly the longer they don't move. This helps show avatars who are either fixed in their positions on a discussion question, or who might be idle; other visualizations help disambiguate between those two states. When an avatar moves, their dwell indicator shrinks slowly, and their path is mapped out with a transparent trail. Combined, these visualizations can tell simple stories about someone who had long disagreed on a point and then was swayed and shifted to being in the middle after a particularly convincing argument was made. On the gradient floor, the average position of avatars on the floor is displayed, to demonstrate a sort of visual consensus. In the space above the meeting room, this average is recorded over time by bars that float slowly up above the meeting room. Much like individual movement is recorded by trails on the meeting room floor, the history of the group average feeling is recorded by these floating bars in the sky. Chat messages are recorded similarly. Each time an avatar talks, a small box appears above them and floats up with the average vote bars. This history of the meeting helps participants reflect on meeting process by showing, similar to \citet{second_messenger}, relative participation. It adds to that work, though, by also showing where in the space people where when they were talking, which effectively adds socially significant metadata to the participation visualization.


% do I need to say anything about evaluation here? or can I just end it here?
% I could do a critique of why virtual worlds aren't effective here, but I kinda don't really feel like I have space for it. 

\subsection{backchan.nl}

Face-to-face encounters, as described in the introduction, have a number of inconvenient biases. One venue where these biases rapidly become visible is when audiences ask questions of presenters and panelists at conferences. Many people might not be comfortable asking questions in front of an audience; those that do feel comfortable asking questions are often asking questions relevant only to themselves, not to the audience as a whole. Even if someone aspires to ask a question that is relevant to a broad section of the audience, how would they know? Furthermore, people tend to ask questions about the content towards the end of a presentation because it's foremost in their mind; recalling questions they might have had from early in the presentation is difficult.

\emph{backchan.nl} addresses these problems by providing a shared space for asking questions and voting on other people's submitted questions. Votes can be either positive or negative. The top eight questions are projected on a display in the space that is visible to the audience, presenters, and moderators. Audience members submit their questions and votes using a web interface that is accessible on either a traditional web interface or an interface sized for mobile devices. Top questions are determined by a ranking algorithm that rewards recent, positive votes. This makes sure that the top posts view doesn't stagnate over long periods. 

Although it was initially designed for a specific panel-based conference, \emph{backchan.nl} has subsequently been deployed in a wide variety of situations. Over the systems almost three year lifetime, more than $400$ events have been created, with over $1200$ meetings. Almost $12000$ unique users have posted or voted in a \emph{backchan.nl} event, creating more than $15000$ posts with almost $60000$ votes. Its broad use has pushed it far beyond its original question-oriented focus. In our study on its use in a pair of early conferences to use \emph{backchan.nl} \citet{backchannl_chi}, we found that although we intended it to be used for questions, it quickly evolved to support content sharing, advertising other backchannels, technical support, as well as communicating about shared issues like temperature, audio problems, or lost items.

My main finding in the \emph{backchan.nl} work was focused on the importance of a shared display. In terms of the themes I introduced at the beginning, we were focusing on the role of grounding in communication channels. The major problem with past work in this area was, in my view, that backchannels tended to be invisible in the presentation space itself. This led to a number of problems. Conversations tended towards criticism and snarky-ness, not constructive commentary, seen most vividly in an ETech talk from danah boyd in 2009 \citet{danah_etech_commentary} or the so-called ``Twitter revolt'' at a Mark Zuckerburg interview in 20XX. \citet{figure_this_out}. With \emph{backchan.nl}, I found this sort of discourse was quite rare and when it did occur it tended to be voted down quickly. The grounding created by the shared display was also important for created a sense of shared space that made it a useful venue for non-question activities like advertising. Grounding also gives it legitimacy because presenters, conference producers, and the audience can't ignore it. The presence of public communication means it must be addressed by people present in the room, which empowers audiences in ways that non-public backchannels have not been able to.

\section{Channels in Practice: Tin Can}

\subsubsection{Tin Can Classroom}

\subsubsection{Tin Can Meetings}



\section{Contributions}

\section{Timeline}



\bibliography{sample-handout}
\bibliographystyle{plainnat}



\end{document}
