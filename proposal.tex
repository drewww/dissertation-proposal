
\section{Introduction}

% There are a few different strategies we can go with here. We could attack the beyond being there problem, and go for the "thinking about face to face" bit. Alternatively, a more general plea for shifting attention back to synchronicity - asynchronous has its issues, and we still (rightfully) privilege real time encounters. The motivating metaphor here being the weirdly object-centric models of asychronous encounters, not people-centric models.

% Some other related themes:
% - stages
% - channels
% - attention
% - text v audio
% - power relationships / confidence?
% - ???

% The core question is what do my systems encompass and lend themselves more to. Clearly they're all synchronous experiences, not async. So that's fine. 

% Okay, here's what we'll do - leave the sync/async pendulum stuff to the background section as a way of framing the history of system design in this space. Focus on the beyond being there story for the introduction.

\section{Background}

% We'll need to do a little organization here. Obviously, farm the references from the Tin Can Edu CSCW paper + backchan.nl paper. We'll need more, ofc, but it's a start. I suspect there will be some ways to separate out sytems that allow communication versus those that simply reflect on a main channel. Maybe it's about whether the system itself is a single channel, multi channel, main channel or side channel? 


\section{Systems in Practice}

\section{Contributions}

\section{Timeline}

